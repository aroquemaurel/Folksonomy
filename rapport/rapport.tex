\documentclass[12pt,a4paper,openany]{book}
\usepackage{lmodern}
\usepackage[table]{xcolor}
\input{/home/aroquemaurel/cours/includesLaTeX/couleurs.tex}

\usepackage[utf8]{inputenc} \usepackage[T1]{fontenc}
\usepackage[francais]{babel}
\usepackage[top=1.7cm, bottom=1.7cm, left=1.7cm, right=1.7cm]{geometry}
\usepackage{verbatim}
\usepackage[urlbordercolor={1 1 1}, linkbordercolor={1 1 1}, linkcolor=vert1, urlcolor=bleu, colorlinks=true]{hyperref}
\usepackage{tikz} %Vectoriel
\usepackage{listings}
\usepackage{fancyhdr}
\usepackage{multido}
\usepackage{float}
\usepackage{amssymb}
\usepackage{longtable}
\usepackage{wrapfig}

\newcommand{\titre}{Conception d'une application de folksonomies}

\newcommand{\pole}{}
\newcommand{\sigle}{siaweb}

\newcommand{\semestre}{4}

\input{/home/aroquemaurel/cours/includesLaTeX/listings.tex}
\input{./docsProjet.tex}
\input{/home/aroquemaurel/cours/includesLaTeX/remarquesExempleAttention.tex}
\input{/home/aroquemaurel/cours/includesLaTeX/polices.tex}
\input{/home/aroquemaurel/cours/includesLaTeX/affichageChapitre.tex}
\let\pagebreakORIG\pagebreak
\let\clearpageORIG\clearpage
\let\cleardoublepageORIG\cleardoublepage

\ifx \removepagebreak \undefined
\newcommand{\removepagebreak}{\renewcommand{\pagebreak}{}\renewcommand{\clearpage}{}\renewcommand{\cleardoublepage}{}}
\fi

\ifx \restorepagebreak \undefined
\newcommand{\restorepagebreak}{\renewcommand{\pagebreak}{\pagebreakORIG}\renewcommand{\clearpage}{\clearpageORIG}\renewcommand{\cleardoublepage}{\cleardoublepageORIG}}
\fi
\newcommand{\pfp}{\texttt{pfp}}

\newcommand{\ifp}{\texttt{if}}
\newcommand{\elsep}{\texttt{else}}

\makeatother
\includeonly {
}
\newcommand{\bootstrap}{\textit{bootstrap}}
\begin{document}
	\setcounter{tocdepth}{2}
	\setcounter{secnumdepth}{3}
	\removepagebreak
	\maketitle
	\newpage
	\chapter*{Avant-propos}
	Ce dossier comporte les différentes de réalisation d'une application Web de folksonomies, site permettant de partager des URL et
	de les associer à des tages.  
	Il à été conçut par Antoine de \bsc{Roquemaurel} dans le cadre du module \textit{Systèmes d'Information et Application Web} de la L2 Informatique de l'université Toulouse III -- Paul Sabatier.

	\section*{Tester le projet}
	L'archive que vous avez reçus est organisée comme ceci: 
	\begin{description}
		\item[folksonomies/] Contient tous les fichiers du Site Web. 
		\item[rapport.pdf] Le présent rapport que vous êtes en train de lire
	\end{description}

	Afin de tester le projet, vous pouvez avoir accès au site web fonctionnel directement en ligne à l'adresse
	\url{http://dev.joohoo.fr/dev/folksonomies/}. 
	Il est également possible d'utiliser notre code source avec votre propre serveur web et votre base de données, pour cela les paramétrage des accès à la base de données 
	sont présent dans le fichier \texttt{folksonomies/database/connect.php}. Toutes les adresses web étant en relatifs, aucun problème ne devrait avoir lieu.

	Ce site Web utilisant des fonctionnalités de \bsc{HTML}5 et \bsc{CSS}3, il est recommandé d'utiliser un navigateur récent. Ce site à été développé sous
	Google Chrome, ainsi l'affichage sera optimal sur ce navigateur, cependant il devrait s'afficher correctement sur les autres.

	Il est possible de créer le schéma de la base de données grâce au menu Base de données $\rightarrow$ Création de la base.
	\vfill
	\footnotesize Rédigé le \today{} par Antoine de \bsc{Roquemaurel}.
	\restorepagebreak
	\tableofcontents
	\chapter{Les besoins de l'application}
	Ce projet consiste en la création d'un site web utilisant les technologies \bsc{HTML}, \bsc{CSS}, \bsc{PHP} et la base de données \bsc{MySQL}
	permettant de gérer des tags sur des sites web. Ces tags permettent de retrouver facilement des URL, d'effectuer des recherches
	en fonction de tags, \ldots

	Pour cela, un modèle de base de données nous à été fournis, mon application ayant été légèrement améliorée afin de pouvoir gérer
	une multitude d'utilisateurs, celui-ci à été légèrement modifié. Figure \ref{fig:redmine} est présent le nouveau modèle. 
	\begin{figure}[H]
		\centering
		\includegraphics[width=10cm]{screens/mcd.png} 
		\caption{Modèle conceptuel de données de l'application}
		\label{fig:redmine}
	\end{figure}

	\begin{description}
		\item[Insertion d'un nouveau document] 
	Le site doit permettre d'ajouter un nouveau document, et de le lier à des tags. Si l'adresse du site existe déjà dans la base de
	données, celle-ci n'est pas ajoutée de nouveau, seuls les nouveaux tags sont ajoutés.
		\item[Lister les documents enregistrés sur le site] 
	Il doit être possible de lister tous les documents présent dans la base de données, ainsi que les tags associés. Il est également
	possible de savoir quels utilisateur ont enregistrés un document.
		\item[Afficher un nuage de tag] 
	L'application doit pouvoir afficher un << tag cloud >> ou nuage de tag. Les différents tag s'affiche d'une taille différente et
	d'une couleur plus foncée en fonctions de leur importance, c'est-à-dire du nombre de document liés à ce tag. 
		\item[Afficher tous les docuemnts liés à un tag donné] 
	Il doit être possible d'afficher tous les documents étant liés à un tag donné.
		\item[Rechercher un document] 
	On peut rechercher un document en fonction d'un ou plusieurs tags. Si plusieurs tags sont rensignés, la rechercher retournera
	tous les documents étant associés à au moins un des tags.
	\item[Inscription d'un nouvel utilisateur] 
	Il est également possible de s'inscrire afin d'utiliser le système, pour cela un pseudo, une adresse e-mail et un mot de passe sont
	demandés, les mots de passes sont cryptés dans la base de données. L'inscription d'un utilisateur lui permet ensuite d'ajouter
	de nouveaux documents, de leur donner une description personnalisée et de lister tous les documents lui appartenant.
	\item[Afficher les documents enregistrés par un utilisateur] 
	Enfin, un utilisateur peut afficher uniquement les doc\-uments lui appartenant.
	\end{description}

	\chapter{Organisation du travail}
	Pour ce projet, j'étais seul, ainsi cela fut plus simples que lors du projet précédent où nous étions deux à travailler dessus. 

	Cependant, afin de bien organiser mon travail, de n'oublier aucune fonctionnalités et de ne rien perdre en cas de problème
	technique, j'ai choisis pour ce projet d'utiliser un outil de gestion de projet et un logiciel de versionnement de la même
	manière que le projet précédent.
	\section{Un outil de gestion de projet : Redmine}
	\begin{figure}[H]
		\centering
		\includegraphics[width=18cm]{screens/redmine.png} 
		\caption{Affichage des demandes dans Redmine}
		\label{fig:redmine}
	\end{figure}
	Pour le projet, j'ai utilisé \textit{Redmine}, une plateforme web de gestion de projet (Cf figure \ref{fig:redmine}). Elle m'a permis 
	de simplifier le travail, et de ne rien oublier.

	En effet, je pouvais créer des tâches, signaler qu'elles sont en cours/terminés/en tests, leur donner des dates limites,
	etc\ldots 

	\section{Un logiciel de versionnement : Git}
	Afin de limiter les problèmes liés à une perte de données, j'ai utilisé un logiciel de versionnement Git. Il a deux intérêt, tout
	d'abord, je pouvais travailler de plusieurs ordinateurs en parallèle sur le projet sans me soucier de fusionner le travail\footnote{À condition de ne pas travailler sur deux lignes de code identiques}.

	D'autre part, tous les logs étant enregistrés, je pouvais voir l'avancée du projet. 

	Enfin, toutes les modification sont stockées sur le serveur, ainsi en cas de problème, il est très facile de revenir à la version précédente ou même de
	comparer deux versions afin de voir les changements et de comprendre rapidement pourquoi une fonctionnalité à régressé. 

	\chapter{Implémentation et conception}
		La conception du projet est l'étape la plus importante, ainsi j'ai préféré repenser à ma manière le projet, 
		je suis partis sur une architecture respectant le patron de conception MVC\footnote{Modèles Vue Contrôleur ou
		\textit{pattern MVC} : Model View Controller en anglais}, c'est à
		dire que toute la partie affichage (notamment le \bsc{HTML}), Modèle (c'est-à-dire les requêtes \bsc{SQL}) sont séparés, 
		c'est le Contrôleur qui dirige le modèle et la vue, c'est ainsi le \textit{chef d'orchestre}.
		\section{Mise en forme : les vues}
			Afin d'avoir un site web élégant sans passer beaucoup de temps à réinventer la roue en \bsc{CSS}, j'ai choisi d'utiliser un
			framework\footnote{Kit de composants logiciels structurels, qui sert à créer les fondations ainsi que les grandes lignes de tout ou d’une partie d'un
			logiciel (architecture). --- D'après \textit{wikipédia}} CSS
			appelé \bootstrap{}. Celui-ci nous permet d'avoir une mise en forme sobre très rapidement.  La mise en forme est organisé comme suit:

			\subsection{Le CSS} Le \bsc{css} est présent dans le dossier \texttt{style/css}, dedans est présent d'une part le
			\bsc{css} fourni par \bootstrap{}, auquel je n'ai pas touché, d'autre part, mon \bsc{css} qui surcharge certains affichage de \bootstrap{}, celui-ci est dans \texttt{style/css/style.css}.
			\lstinputlisting[language=CSS, caption=Code CSS]{code/css.css}
			\subsection{Le JavaScript}
			Le JavaScript quant à lui est dans le dossier \texttt{style/js/}, cependant je n'ai personnellement pas modifié de
			JavaScript, je me suis servis de celui existant pour quelques fonctions de base, comme l'affichage d'une boite de
			dialogue ou des messages d'informations pouvant être masqué à l'aide d'un clic.

			\bootstrap{} utilise le framework JavaScript Jquery.
				
		\subsection{Le \bsc{HTML}}
		Le \bsc{html} correspond aux vues, tous les fichiers contenant du HTML sont dans le dossier \texttt{views/}, les formulaires 
		eux sont dans le dossier \texttt{views/forms}.
		
		Les vues peuvent ne comporter que du \bsc{html}, comme c'est le cas dans les formulaires, mais ils peuvent également comporter des instructions simples
		de \bsc{php} comme des boucles ou des conditions.

		Le dossier \texttt{views} contient également les fonctions d'affichage des documents ou une fonction permettant d'afficher une alerte. 
			\lstinputlisting[language=PHP, mathescape=false, caption=Fonction d'affichage d'une liste de documents]{code/documents.php}
		\section{Connexions à la base de données : Les modèles}
		L'application comporte trois modèles : un pour les documents, un pour les termes, et un pour les utilisateurs. Ces trois fichiers contiennent des fonctions effectuant des
		requêtes et retournant un tableau avec les attributs, ceci afin de bien séparer le \bsc{sql} du \bsc{php}.

		Également présent dans le dossier \texttt{database}, le fichier \texttt{connect.php} contient les identifiants et les
		instructions de connexion à la base de données. Le fichier \texttt{database.php} quant à lui contient les requêtes de
		création et suppression de la base.

		\lstinputlisting[language=SQL, mathescape=false, caption=Requête sélectionnant les documents d'un utilisateur donné ]{code/sql.sql}
		\lstinputlisting[language=PHP, mathescape=false, caption=Fonction d'insertion d'un document]{code/sql.php}
		\section{Liaison des vues aux modèles : Les contrôleurs}
		Chaque contrôleur correspond aux pages que verra un utilisateur lambda dans l'URL, les contrôleurs sont en général assez
		simples, ils font appels au modèle puis font une inclusion de la vue, ceux sont eux qui feront les éventuels vérifications
		sur des paramètres, sur une connexion pour l'insertion de document etc\ldots
		\lstinputlisting[language=PHP, mathescape=false, caption=Contrôleur de l'insertion d'un document]{code/controleur.php}
		Comme dit plus haut, c'est relativement simple, on inclus les vues et fonctions nécessaires, on vérifie qu'on a le droit
		d'être ici, on créer le tableau de mots clés, puis on insère le tout en faisant appel au modèle.

	\chapter{Résultats obtenus}
	Dans cette partie, nous allons regrouper quelques captures d'écrans montrant les résultats obtenus sur notre application Web.
	L'application est répartie en 5 fonctionnalités principales : 
	\begin{itemize}
		\item Nuage de tags
		\item Recherche
		\item Liste des documents d'un utilisateur
		\item Affichage de << Mes documents >> 
		\item Création et suppression de la base de données
	\end{itemize}
	\section{Nuage de tags et recherche}
	\begin{figure}[H]
		\centering
		\includegraphics[width=18cm]{screens/index.png} 
		\caption{Nuage de tags et recherche} 
		\label{fig:index}
	\end{figure}
	Comme demandé dans les besoins, notre applications permet d'afficher un nuage de tags, celui-ci est visible sur la première
	partie de la figure \ref{fig:index}. Les couleurs et les tailles des mots clés varient en fonction de leur nombre d'occurrence
	dans les différents documents.

	Une recherche est également possible sur la page d'accueil, celle-ci se trouve en dessous du nuage de tag sur la figure
	\ref{fig:index}. Il est possible de renseigner plusieurs mots clés différents, ceux-ci doivent être séparé par des espaces dans
	le champ de recherche.\\
	Le résultat de la recherche s'affiche en dessous du champ de recherche.

	\remarque{Ces deux fonctionnalités sont disponible dans être identifiés, tout le monde peut donc y avoir accès}
	\section{Liste de tous les documents}
	Il est possible d'afficher la liste de tous les documents présents sur le site. Ceux-ci s'afficheront de la même manière que
	lors de l'affichage dans la recherche. Dans cette liste le clique sur un tag, renvoie sur un affichage de tous les documents
	associés à ce tag, le clique sur un utilisateur affiche tous les documents de cet utilisateur, enfin le clique sur le nom du
	document correspond à l'adresse web enregistrée en base de données.

	Cette liste est disponible figure \ref{fig:listeDocs}.
	\begin{figure}[H]
		\centering
		\includegraphics[width=18cm]{screens/documents.png} 
		\caption{Liste de tous les documents} 
		\label{fig:listeDocs}
	\end{figure}
	\remarque{La liste de l'intégralité des documents du site est disponible à tout le monde, ainsi l'authentification n'est pas
	nécessaire.}

	\section{Liste des documents d'un utilisateur}\label{docUser}
	Comme le montre la figure \ref{fig:docUser} il est possible d'afficher les documents enregistrés par un utilisateur. Ces
	documents n'affichent pas la description mise par l'utilisateur, cette information étant considérée confidentielle.
	\begin{figure}[H]
		\centering
		\includegraphics[width=18cm]{screens/documentUser.png} 
		\caption{Liste des documents d'un utilisateur}
		\label{fig:docUser}
	\end{figure}
	\section{Affichage de << Mes documents >>}
		Lorsque l'on est connecté, on a la possibilité d'afficher ses propres documents. Ceux-ci s'affiche de la même manière que
		section \ref{docUser} cependant, une information supplémentaire est affichée: la description mise par l'utilisateur. 
	\begin{figure}[H]
		\centering
		\includegraphics[width=18cm]{screens/mesDocs.png} 
		\caption{Liste de mes documents} 
		\label{fig:index}
	\end{figure}
	\section{Création et suppression de la base de données}
	Il est également possible de recréer ou de supprimer la base de données. Supprimer une base inexistante, ou créer une base déjà
	existante ne créera pas de bug, cependant l'action ne s'effectuera pas.\footnote{Utilisation de l'instruction mysql \texttt{if
	exist}}.

	Pour éviter les erreurs maladroites, une boite de dialogue s'affiche en cas de demande de suppression de la base de données,
	comme le montre la figure \ref{fig:suppression}.
	\begin{figure}[H]
		\centering
		\includegraphics[width=18cm]{screens/suppression.png} 
		\caption{Demande de suppression de la base de données} 
		\label{fig:suppression}
	\end{figure}
	\appendix
	\listoffigures
	\removepagebreak
	\vfill
	\lstlistoflistings
	\vfill
	\restorepagebreak
\end{document}

